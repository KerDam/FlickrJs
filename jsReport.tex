\documentclass[a4paper, 11pt]{article}
\usepackage[utf8]{inputenc}
\usepackage[french]{babel}
\usepackage[T1]{fontenc}

\begin{document}
\large\textbf{Projet JQuery}  \textbf{Damien Farce, Augustin Lacour}

\section*{Présentation du projet}
Le but du projet est de créer une interface WEB permettant d'afficher des photos provenant de Flickr en lien avec une ville. L'interface doit offrir les fonctionnalitées suivantes:
\newline
\begin{itemize}
    \item \textbf{L'autocomplétion à partir d'une application PHP.}
    \item \textbf{Récupération des photos en fonction de la ville qui est saisie.}
    \item \textbf{2 onglets qui permettent l'affichage en tableau et l'autre en liste.}
    \item \textbf{L'affichage des photos sur une carte si les informations de géolocalisation sont disponible.}
    \item \textbf{D'afficher seulement les photos qui ont été prises à partir d'une certainne date. }
    \item \textbf{Offrir le choix à l'utilisateur du nombre de photos qu'il veut afficher}
\end{itemize}

\section*{Présentation des outils}
Pour réaliser ce projet nous devons utiliser les technologies Web classiques HTML/CSS/JS et plus particuliérement JQuery, un framework Javascript qui permet de faire des requêtes réseaux asynchrone et fourni d'autre outils tels que la gestion de popups, de table (trie) etc. 
\newline
\newline
De plus pour la gestion de la carte nous avons préféré utilisé Leaflet par rapport à Google Maps car elle permet une gestion plus souple des cartes en de permettre d'utiliser des cartes open-sources.
\newline
\newline
Pour la génération de la documentation nous avons utilisé NaturalDocs car son utilisation simple et sa compatibilité avec plusieurs languages en font un outil puissant. 

\subsection*{L'api Flickr}
L'api (application programmation interface) de Flickr s'utilse en faisant des requêtes HTTP sur un serveur, les résultats sont du texte formatées en JSON ou en XML, les informations qui sont contenue dans le JSON permet de récupérer les URLS pour les images ainssi que d'autre informations intérréssantes.
\newline
\newline
Il y a 2 façon d'intérargir avec l'api:
	\begin{itemize}
            \item \textbf{Sans clé: permet seulement de rechercher des photos qui sont public, l'accés au informations est réduit et il y a une limites de photos.}
            \item \textbf{Avec une clé: Donne accés à des requêtes plus avancées nottament de sélectionner les photos à partir d'une certainnes date et d'avoir les informations de géolocalisation.}
	\end{itemize}
\newline 
Nous tenons à remercier Adrien Gennevois qui nous a préter sa clé d'API.

\section*{Réalisation du projet}

	\subsection*{Page HTML }
        La préssentation de la page HTML est un simple formulaire avec:
        \newline
        \begin{itemize}
            \item \textbf{La sélection de la ville avec autocomplétion via des requêtes AJAX}
            \item \textbf{Le choix du nombre de photos à afficher}
            \item \textbf{Le choix de la qualitée}
            \item \textbf{Le choix d'une date (optionel)}
        \end{itemize}
        \newline
        Une fois la recherche lancée:
        \newline
	\begin{itemize}
            \item \textbf{Une carte est affichée si des informations de géolocalisation sont disponible}
            \item \textbf{Le premier onglet présente les images les unes à la suite des autres avec aucunne informations particulière et lorsqu'on click sur une photo une fenêtre modale s'affiche}
            \item \textbf{Le second onglet affiche les résultat dans une table qui permet de faire des tris selon différents paramétres tels que la date, le titre etc...}
	\end{itemize}
	\subsection*{Le traitement}
	Une fois que l'utilisateur a lancé le traitement en appuyant sur le boutton de recherche, une réquête Ajax va intérroger l'API flickr pour récupérer les informations relatives au photos concernant la ville. Le résultat obtenu est du JSON qui est ensuite parsé, pour chaque élément JSON:
        \newline
        \begin{itemize}
            \item \textbf{Ajout d'une balise <img> dans la table et dans la vue en liste avec comme href l'url de l'image}
            \item \textbf{Ajout d'un click listener à la balise pour gérer l'affichage de la fenêtre modale contenant de des informations suplémentaire lors du click.}
            \item \textbf{Ajout d'un marker sur la carte si les données de géolocalisation existent, ce marker contient le nom de la photo ainssi que la photo en miniature}
        \end{itemize}
	\subsection*{Les Problèmes}
        Le sujet n'était pas trés clair s'il était nécéssaire d'avoir une clé d'API pour faire le projet, ce n'est donc que lors de la gestion de la carte que nosu nous sommes rendu compte de l'importance de la clé d'API pour avoir les données de géolocalisation, ors le format des requêtes et des résultats n'est pas éxactement le même, il a donc fallut adapter le traitement du JSON.

\section*{Conclusion }
L'application que nous avons réalisé proposent toutes les fonctionnalitées qui était attendu dans le sujet + la gestion de la carte qui était en bonus et le choix de la qualité qui malgré sa faible utilité a le mérite d'être là.
\end{document}

